\section{Zusammenfassung und Fazit}
\label{sec:zusammenfassung}
In dem vorliegenden Projekt ist das Ziel die Segmentierung von Gewässern auf Satellitenbildern.
Vorerst musste der Datensatz, bestehend aus etwa $\SI{58000}{}$ Satelliten- und Maskenbildern von Gewässern erzeugt und verarbeitet werden.
Das Maskenbild beinhaltet nur Gewässer, also genau das was der Algorithmus segmentieren soll.
\\
Das Problem wird über zwei Wege angegangen.
Zum einen wird ein tiefes neuronales Netz mit Convolutional Ebenen (CNN) verwendet.
Durch eine besondere Anordnung der Ebenen (U-Net), konnte ein effizientes Netz mit einer geringen Parameteranzahl zur Segmentierung erzeugt werden.
Die wichtigsten Hyperparameter wurden mit einer zufälligen Gittersuche optimiert.
\\
Eine andere Methode, der sog. Random Forest segmentiert Satellitenbilder anhand von Farbwerten und Gradienten der Pixel.
Das Training ist sehr ineffizient und konnte deshalb nur auf einem Teil des Datensatzes durchgeführt werden.
\\
\\
Rückblickend kann man sagen, dass der Random Forest trotz der erzielten Genauigkeit von über $\SI{89}{\percent}$ für diese Problemstellung nicht geeignet ist.
Dieser verarbeitet die Pixel separat und kann somit keine zusammenhängende Gebiete erkennen.
Diese hier notwendigen Verbindungen zwischen den Pixeln, kann ein tiefes neuronales Netz, wie das CNN herstellen.
\\
Aufs Ganze gesehen hat das CNN mit einer Genauigkeit von über $\SI{93}{\percent}$ das vorliegende Problem erfolgreich lösen können.
Die Ergebnisse übertrafen unsere (aufgrund des teils fehlerhaften Datensatzes) niedrige Erwartungshaltung um ein Vielfaches.