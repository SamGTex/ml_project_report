%\section{Motivation}
%\label{sec:Motivation}
Maschinelles Lernen wird zunehmend relevanter und hat bereits heute große Anwendungsgebiete.
Anfänglich wurde künstliche Intelligenz (KI) in der Wissenschaft entwickelt, revolutioniert und angewendet.
Heutzutage ist KI kaum aus dem Alltag wegzudenken und ein wichtiger Bestandteil der Datenanalyse geworden.
\\
Zum Beispiel basieren Vorschläge von Anzeigen/Videos/Produkten auf etlichen 'großen' Internetseiten, auf Vorhersagen von KI.
Die KI bekommt dabei jegliche Nutzerinformationen als Input, wie die zuletzt besuchten Seiten / geschaute Videos / gekaufte Produkte und generiert ein Interessensprofil, auf dessen Grundlage eine Vorhersage (Output) getroffen wird.
Auch Anwendungen wie Spracherkennung, Autokorrektur bzw. Vorhersage des nächsten Worts oder Bilderkennung wurden mithilfe von KI revolutioniert und wären ohne kaum realisierbar gewesen.

%unnötig
%Maschinelles Lernen kann in die drei große Gebiete 'überwachtes', 'unüberwachtes' und 'bestärkendes' Lernen unterteilt werden.
%Wird die KI für korrekte Entscheidungen belohnt (z.B. fürs schlagen einer Schachfigur), so spricht man vom sog. 'bestärkenden' Lernen.
%Beim 'unüberwachten' Lernen werden in Daten (Input) Muster/Strukturen erkannt, die sich vom Rauschen der Daten unterscheidet.
%
Das Projekt befasst sich mit dem sog. 'überwachten Lernen'.
Hierbei werden zu den Input-Daten $X$, bereits die Ergebnisse $Y$ angegeben.

\section{Problemstellung}
