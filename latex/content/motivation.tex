%\section{Motivation}
%\label{sec:Motivation}
Maschinelles Lernen wird zunehmend relevanter und hat bereits heute große Anwendungsgebiete.
Anfänglich wurde künstliche Intelligenz (KI) in der Wissenschaft entwickelt, revolutioniert und angewendet.
Heutzutage ist KI kaum aus dem Alltag wegzudenken und ein wichtiger Bestandteil der Datenanalyse geworden.
\\
Zum Beispiel basieren vorgeschlagene Anzeigen/Videos/Produkte auf etlichen \enquote{großen} Internetseiten, auf Vorhersagen von KI.
Die KI bekommt dabei jegliche Nutzerinformationen als Eingabe, wie die zuletzt besuchten Seiten / geschaute Videos / gekaufte Produkte und generiert ein Interessensprofil auf dessen Grundlage die Vorhersage getroffen wird.
Auch Anwendungen wie Spracherkennung, Autokorrektur oder Bilderkennung wurden mithilfe von KI revolutioniert und wären ohne diese kaum realisierbar.
\\
%unnötig
%Maschinelles Lernen kann in die drei große Gebiete 'überwachtes', 'unüberwachtes' und 'bestärkendes' Lernen unterteilt werden.
%Wird die KI für korrekte Entscheidungen belohnt (z.B. fürs schlagen einer Schachfigur), so spricht man vom sog. 'bestärkenden' Lernen.
%Beim 'unüberwachten' Lernen werden in Daten (Input) Muster/Strukturen erkannt, die sich vom Rauschen der Daten unterscheidet.
%
Das Projekt befasst sich mit dem sog. überwachten Lernen.
Hierbei erhält das Netzwerk zu den Eingangsdaten, bereits die Ergebnisse bzw. Ausgangsdaten und bildet beim Training die Gesetzmäßigkeiten nach.

\section{Problemstellung}
Bereits seit $1957$ befinden sich die ersten Satelliten auf einer Umlaufbahn um die Erde.
Derzeit befinden sich $4084$ (30.04.2021) Satelliten im Betrieb und kreisen im Orbit um die Erde.\cite{ucsusa}
\\
Einige der Satelliten, sog. Erdbeobachtungssatelliten nehmen dabei Bilder der Erde auf, um unter anderem basierend auf diesen Karten zu erstellen oder Veränderungen der Erdoberfläche festzustellen.
Beispielweise können Klimaforschende so die Entstehung neuer Gewässer bzw. ausgetrocknete Gewässer erkennen und analysieren.
\\
In diesem Projekt soll es genau darum gehen, die Erkennung und Segmentierung von Gewässern auf Satellitenbildern.
\\

%\fcolorbox{red}{white}{}
\fbox{\parbox{\dimexpr\linewidth-2\fboxsep-2\fboxrule\relax}{\centering Wo befinden sich auf einem gegebenen Satellitenbild \\ innerhalb Europas größere Wasserflächen?}}%
\\
\\
Um allein die Fläche Europas vollständig abzudecken, wären Millionen Satellitenbilder erforderlich.
Das liegt daran, dass nur durch einen hinreichend kleinen Zoom-Faktor mittelgroße Gewässer erkennbar sind.
Aufgrund der Kapazität, wird sich zum einen auf das europäische Festland begrenzt.
Zum anderen sollen nur größere Gewässer und keine Bäche, Teiche oder Ähnliches erkannt werden.
\\
Die Auswertung von Satellitenbildern in großer Zahl ist per Hand mühsam und kostet viel Zeit.
Für solch große Datenmengen bietet sich eine Automatisierung mithilfe einer KI an.
Hier können in wenigen Sekunden tausende Satellitenbilder ausgewertet werden.